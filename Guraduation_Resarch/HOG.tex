HOG特徴とは、LBP特徴と同じ画像特徴の一つで入力画像の輝度勾配ヒストグラムを特徴するものである。
HOG特徴の求め方は、まず入力画像にグレースケール化を行った後、いくつかのピクセルが集まってできたセルと
セルがいくつか集まってできたブロックに分ける。今回の実験では1セル20$\times$20ピクセルとし、1ブロック3$\times$3のセルの集合とした。
次に1ブロックに含まれるセルごとの強度と輝度勾配を計算する。
強度と輝度勾配を求めるためにCannyのエッジ検出と同様に画像に対してx方向とy方向に対してSobelフィルタをかけていく。
強度と輝度勾配の式は(2)、(3)式を使用する。
その後、セル毎にセル内の輝度勾配のヒストグラムを作成(図\ref{HOG})し、それをブロック単位で連結しHOG特徴とする。


HOG特徴は、画像の特定領域内の輝度勾配を用いるので物体の形状を表現することができ、SVMと使うことで人物検出や車両の検出などに用いられている。

今回の実験でHOG特徴を使用した理由としては、風景画像には建物や山のような目印となるものほかに道路の白線等の直線といった形状情報を
含んでいる。そういった、情報をより強調することができるのがHOG特徴である。前処理でHOG特徴を使用することで
テンプレートマッチングを行った際に一致している時と一致していない時での類似度に差が出ると予想した。