LBP特徴(Local Binaly Ptterns)とは、画像特徴量の一つで注目画素とその周りの輝度のパターンを特徴とするものである。
LBP特徴は、まず入力画像にグレースケール化を行う。その後、注目画素とその周り八近傍に存在する画素の画素値を比較する。
比較した結果注目画素よりも近傍に存在する画素の方が画素値が大きいのであればば1、小さければ0とする。
その後、注目画素に対して12時方向に存在する画素から反時計回りに1週0,1の値を取得する。
取得したその値を2進数に置き換えて得られた値が、注目画素のLBP特徴となる。
例えば、図\ref{LBP}のように中心の画素値が144の場合、八近傍に存在する画素値をそれぞれ144よりも大きければ1、小さければ0とする。
こうして0と1に振りなおされた八近傍の中から注目画素の12時方向から反統計周りに値を取得していく。
今回の例でいえば、12時方向の値は1でありここから反時計回りに取っていくと、10100100が得られる。
10100100を10進数に直すと164となり、これが注目画素のLBP特徴値となる。

LBP特徴は、注目画素とその周りの八近傍の画素値との相対的な輝度差によって得られるため、
画像全体の明るさやコントラストが変化したとしてもLBP特徴は変化しにくく、
画像に対して回転などの線形変換が行われたとしても2進数のパターンがシフトするだけで
全体的なLBP特徴の変化は少ない。

今回の実験でLBP特徴を使った理由としては、風景画像には目印となる建物や山といった変化の少ない情報のほかに、天候や時間帯による明度の変化など
の常に変化する周りの環境の情報も含まれる。前処理なしで晴れの日の画像と曇りの日の画像でテンプレートマッチングを行った場合、快晴の空と曇りの空で
差が出てしまいマッチングが上手くいかないことがある。そこでLBP特徴を使うことで快晴と曇りのコントラストの差をある程度取ることができ、
さらに画像のスケールを縮小することでさらに差を取ることができる。